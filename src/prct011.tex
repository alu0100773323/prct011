\documentclass{beamer}
\usercolortheme[RGB={122,59,122}]{structure}
\usepackage[spanish]{babel}
\usepackage[utf8]{inputenc}

\usetheme{Madrid}

\title[Presentación con Beamer]{Número PI}
\author[Técnicas Experimentales]{María Pérez Canino}
\date[24-04-2014]{24 de abril de 2014}

\begin{document}

\begin{frame}

  \includegraphics[width=0.15\textwidth]{img/ullesc.eps}
  \hspace*{7.5cm}
  \includegraphics[width=0.16\textwidth]{img/fmatesc.eps}
  \titlepage

  \begin{scriptsize}
    \begin{center}
     Facultad de Matemáticas \\
     Universidad de La Laguna
    \end{center}
  \end{scriptsize}

\end{frame} 


\begin{frame}
  \frametitle{Índice}  
  \tableofcontents[pausesections]
\end{frame}  


\section{Primera Sección}

\begin{frame}

\frametitle{Primera Sección}

En esta presentación veremos diferentes fórmulas que necesitan el número PI

\end{frame}

\section{Segunda Sección}

\begin{frame}

\frametitle{Segunda Sección}

\begin{block}{Ejemplo}
  \begin{itemize}
  \item
  Área del círculo:  $A=\pi r^2$
  \pause

  \item
  Área del cilindro: $A=2\pi r(r+h)$
  \pause

  \item
   Identidad de Euler: $e^{\pi i}+1=0$
  \pause

  \item
  Volumen de la esfera: $V=(4/3)\pi r^3$
  \pause

  \item
   Volumen de un cono: $\frac{r\pi^2h}{3}$
  \pause

  \end{itemize}
\end{block}

\end{frame}

\begin{frame}
  \frametitle{Bibliografía}

  \begin{thebibliography}{10}

    \beamertemplatebookbibitems
    \bibitem[Práctica 11]{practica}  
    Práctica 11, técnicas experimentales. 
    (2014)
    {\small $http://campusvirtual.ull.es/1314/pluginfile.php/197661/mod_resource/content/6/GUIA.pdf$} 

    \beamertemplatebookbibitems
    \bibitem[Comandos LaTeX - Página - Fórmulas - Bibliografía]{comandos}  
    Comandos LaTeX - Página - Fórmulas - Bibliografía 
    (2014) 
    {\small $http://campusvirtual.ull.es/1213m2/pluginfile.php/224421/mod_resource/content/3/TeoriaLaTeX.2.pdf$}


  \end{thebibliography}
\end{frame}

\end{document}